\chapter{Introduction}

\paragraph{Assurance is the key to unlocking autonomy} Over the last decade, there has been an explosion in the capabilities of autonomous systems and artificial intelligence. While autonomous systems are becoming more tightly integrated into our daily lives, there is still a gap between their technological capabilities and their actual impact. The use of autonomous systems today provide just a glimpse of the potential that the broader deployment of autonomy in society has to drive economic growth and productivity. 



However, the growing scale of impact of autonomous systems in society means that the consequences of the failure of these systems are correspondingly large. Guaranteeing safety and reliability for increasingly complex systems only grows more challenging whilst simultaneously becoming more crucial for their adoption in society. For example, not only is it much harder compared to simple systems like robotic vacuums, but also much more important, to convince the world that a self-driving car will not crash in order to see widespread adoption. Studies have shown that user concerns over safety, and more broadly, \emph{lack of trust that the system will meet expectations} remains the most significant barrier to deployment for autonomous systems~\cite{KAUR201887,BEZAI202165,MOLNAR2018319}.  


Broad adoption of autonomous systems can provide solutions for emerging societal issues such as congestion~\cite{LIORIS2017292,8734238}, smart urban planning~\cite{GULSRUD201885,NITOSLAWSKI2019101770}, climate change~\cite{KOLOKOTSA2017101,goddard2021global}, and many others~\cite{DUONG2020355}. It is thus necessary to \emph{provably guarantee} to regulators and build trust with consumers that the risk of failure is minimal. Hence, there is a strong need to develop systematic approaches to verify that autonomous systems will comply with all regulations, keep humans safe, and achieve their missions. \emph{Assured autonomy}, where an autonomous system is formally and provably verified to meet precisely stated objectives, is the key to unlocking the full impact that autonomy can bring to our society. 

Assured autonomy has attracted immense interest from government, industry, and academia. It is being widely studied in various contexts as they relate to autonomy including perception, control, learning, motion-planning and decision-making. The focus of the work in this dissertation is on \emph{assured decision-making}. While this concept will be more formally explained in the following sections, I informally say here that assured decision-making is the problem of guaranteeing that the decisions made by an autonomous system will lead to the specified outcome. In the following sections I will lay out formally lay out how I model the decision-making process of an autonomous agent, and the methods by which we assure the decisions lead to desired outcomes. 


\subsection{What is decision-making?}





Of course, this statement leads to questions like: \emph{what does it mean when an autonomous sysem makes a decision?} and \emph{how does one "specify" an outcome?}. These questions have different answers based on the specifics of the context in which they are being asked. I will 




In order for autonomous systems to become more tightly integrated in society, they will need to be able to handle tasks with increasing levels of complexity. Furthermore, it will be necessary for these systems to interact with non-expert users. As human reliance on autonomy grows, it is also necessary to have \emph{guarantees} on the abilities of these systems to achieve their missions. In this context, we are interested in the following questions:
\begin{itemize}
	\item \emph{How can we specify complex tasks to autonomous agents in a formal but intuitive way with quantitative performance guarantees?}
	\item \emph{How can we provide provable guarantees on the abilities of autonomous agents to carry out these complex tasks?}
	\item \emph{How can users qualitatively tune the behaviour of the agents carrying out these tasks?}
\end{itemize}

\subsection{What is decision-making?}

\subsubsection{Decision-making in the autonomy stack}

\subsubsection{Decison-making vs control}

\subsection{Assurance in decision-making}

\subsection{Structure}


\paragraph{Assured autonomy for urban air mobility} One application area in which autonomy is poised to make a fundamental impact is urban air mobility (UAM). Mass transportation of both passengers and cargo using increasingly autonomous air vehicles in metropolitan areas is an imminent possibility. For such an expansive vision to be realized, there are a significant number of technical challenges to be addressed.  First and foremost is the issue of safety. With projections indicating high-volume use of autonomous aircraft in urban air spaces, it is clear that advances in decision-making for autonomous systems with \emph{assured performance} will play a key role in the advancement and acceptance of UAM. Furthermore, any such technical solution must be able to handle the envisioned massive scale of operations as well as communication and sensing restrictions. Second, in any large-scale operations involving autonomy, it will be necessary for non-expert users to interact with and direct increasingly autonomous systems. However, specifying a complex mission is no trivial task and can often require extensive tuning and expertise.

% Metropolitan areas (other areas more scope)
%%%% Combine is too weak 
%% Learning theory 

\paragraph{Technical contributions} In my research, I have employed an interdisciplinary approach to address the technical challenges of assuring autonomy for large-scale complex systems. Specifically, I formulate techniques that draw from formal methods, learning theory, and controls. The field of formal methods is a powerful tool for providing guarantees on performance and safety. However, it suffers from a lack of scalability, restricting its use in partial-information or multi-agent settings. I have designed algorithms that leverage the guarantees of formal methods with the efficiency of techniques in controls and learning to perform complex missions with provable guarantees in large environments. For example, I have designed algorithms for multi-agent surveillance tasks, large-scale traffic management for UAM, safe learning for multiple agents, and others. Providing high-level mission guarantees in such settings was previously beyond the reach of standard approaches in formal methods due to state-space explosion and lack of automated techniques. In my research I have shown not only theoretical guarantees of safety and performance, but also real-world applicability with high-fidelity simulations in collaboration with industry partners such as NASA and Skygrid, as well as experiments on hardware with Sandia National Labs.

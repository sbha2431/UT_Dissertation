\chapter{Introduction}

\paragraph{Assurance is the key to unlocking autonomy} Over the last decade, there has been an explosion in the capabilities of autonomous systems and artificial intelligence. While autonomous systems are becoming more tightly integrated into our daily lives, there is still a gap between their technological capabilities and their actual impact. The broader deployment of autonomy in society has unprecedented potential to drive economic growth and productivity, however, the consequences of the failure of these systems are correspondingly large. It is thus necessary to \emph{provably guarantee} to both regulators and consumers that the risk of failure is minimal. Hence, there is a strong need to develop systematic approaches to verify that autonomous systems will comply with all regulations, keep humans safe, and achieve their missions. \emph{Assured autonomy} is the key to unlocking the full impact that autonomy can bring to our society. 

In order for autonomous systems to become more tightly integrated in society, they will need to be able to handle tasks with increasing levels of complexity. Furthermore, it will be necessary for these systems to interact with non-expert users. As human reliance on autonomy grows, it is also necessary to have \emph{guarantees} on the abilities of these systems to achieve their missions. In this context, we are interested in the following questions:
\begin{itemize}
	\item \emph{How can we specify complex tasks to autonomous agents in a formal but intuitive way with quantitative performance guarantees?}
	\item \emph{How can we provide provable guarantees on the abilities of autonomous agents to carry out these complex tasks?}
	\item \emph{How can users qualitatively tune the behaviour of the agents carrying out these tasks?}
\end{itemize}

\subsection{What is decision-making?}

\subsubsection{Decision-making in the autonomy stack}

\subsubsection{Decison-making vs control}

\subsection{Assurance in decision-making}

\subsection{Structure}


\paragraph{Assured autonomy for urban air mobility} One application area in which autonomy is poised to make a fundamental impact is urban air mobility (UAM). Mass transportation of both passengers and cargo using increasingly autonomous air vehicles in metropolitan areas is an imminent possibility. For such an expansive vision to be realized, there are a significant number of technical challenges to be addressed.  First and foremost is the issue of safety. With projections indicating high-volume use of autonomous aircraft in urban air spaces, it is clear that advances in decision-making for autonomous systems with \emph{assured performance} will play a key role in the advancement and acceptance of UAM. Furthermore, any such technical solution must be able to handle the envisioned massive scale of operations as well as communication and sensing restrictions. Second, in any large-scale operations involving autonomy, it will be necessary for non-expert users to interact with and direct increasingly autonomous systems. However, specifying a complex mission is no trivial task and can often require extensive tuning and expertise.

% Metropolitan areas (other areas more scope)
%%%% Combine is too weak 
%% Learning theory 

\paragraph{Technical contributions} In my research, I have employed an interdisciplinary approach to address the technical challenges of assuring autonomy for large-scale complex systems. Specifically, I formulate techniques that draw from formal methods, learning theory, and controls. The field of formal methods is a powerful tool for providing guarantees on performance and safety. However, it suffers from a lack of scalability, restricting its use in partial-information or multi-agent settings. I have designed algorithms that leverage the guarantees of formal methods with the efficiency of techniques in controls and learning to perform complex missions with provable guarantees in large environments. For example, I have designed algorithms for multi-agent surveillance tasks, large-scale traffic management for UAM, safe learning for multiple agents, and others. Providing high-level mission guarantees in such settings was previously beyond the reach of standard approaches in formal methods due to state-space explosion and lack of automated techniques. In my research I have shown not only theoretical guarantees of safety and performance, but also real-world applicability with high-fidelity simulations in collaboration with industry partners such as NASA and Skygrid, as well as experiments on hardware with Sandia National Labs.

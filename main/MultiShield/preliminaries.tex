\section{Preliminaries}
\label{sec_prel}

\subsubsection*{Basic notations}
As in the previous chapter, we consider reactive systems with a finite set
$\inp$ ($\out$) of Boolean \emph{inputs (outputs)}.
The input alphabet is
$\ialphabet=2^\inp$, the output alphabet is $\oalphabet=2^O$, and
$\alphabet=\ialphabet \times \oalphabet$. The set of finite (infinite) words
over $\alphabet$ is denoted by $\alphabet^*$ ($\alphabet^\omega$), and
we define $\alphabet^{\infty} = \alphabet^* \cup \alphabet^\omega$.  We will also refer
to words as \emph{(execution) traces}.  We write $|\trace|$ for the
length of a trace $\trace\in \alphabet^*$. For an infinite trace $\trace\in \alphabet^\omega$ we define $|\trace| =\infty$. For $\itrace = x_0
x_1 \ldots \in \ialphabet^\infty$ and $\otrace = y_0 y_1 \ldots \in
\oalphabet^\infty$, we write $\itrace \parallel \otrace$ for the
composition $(x_0,y_0) (x_1,y_1) \ldots \in \alphabet^\infty$.
For $i \in \nats$ and a word $\trace = \sigma_0\sigma_1\ldots \in \awords\Sigma$, we define $\trace[i] = \sigma_i$, and we define $\trace[i,j) = \sigma_i\sigma_{i+1}\ldots\sigma_{j-1}$ if $j \in \nats$ and $\trace[i,j) = \sigma_i\sigma_{i+1}\ldots$ if $j = \infty$.
A \emph{language} is a set $\lang
\subseteq  \alphabet^\infty$ of words.
%We denote the set of all languages as $\langset = 2^{\alphabet^\infty}$.


\subsubsection*{Reactive systems}

Each agent in the multi-agent system, as well as the shield, is a \emph{reactive system} which is defined by a
6-tuple $\proc = (\states,q_{0},\ialphabet,\oalphabet,\delta,\lambda)$,
where
$\states$ is a finite set of states,
$q_{0} \in \states$ is the initial state,
$\ialphabet$ is the input alphabet,
$\oalphabet$ is the output alphabet,
$\delta: \states \times \Sigma_{\inp} \to \states$ is the complete transition function, and $\lambda : \states \times \Sigma_{\inp} \to \Sigma_{\out}$ is the output function.
%
Given an input trace $\itrace = x_0 x_1 \ldots \in \ialphabet^\infty$, a reactive system $\mathcal{\proc}$ produces an
output trace $\otrace = \mathcal{\proc}(\itrace) = \lambda(q_0, x_0)
\lambda(q_1, x_1) \ldots \in \oalphabet^\infty$ with $q_{i+1} = \delta(q_i, x_i)$ for all $i \ge 0$.
The set of words produced by
$\mathcal{\proc}$ is denoted $\lang(\mathcal{\proc}) = \{\itrace \parallel \otrace \in
\alphabet^\infty \mid \mathcal{\proc}(\itrace) = \otrace\}$.

\iffalse
\subsubsection*{Serial composition of reactive systems}
Consider two reactive systems  $\proc = (\states,q_{0},\ialphabet,\oalphabet,\delta,\lambda)$ and $\shield = (\states', q_{0}',\ialphabet\times\oalphabet, \oalphabet,\delta',\lambda')$.
The \emph{serial composition} of $\proc$ and $\shield$ is obtained by feeding the input and output of $\proc$ to $\shield$ and results in a new reactive system $\proc \comp \shield=(\hat{\states}, \hat{q_{0}}, \ialphabet,\oalphabet, \hat{\delta},
\hat{\lambda})$, where
   $\hat{\states} = \states \times \states'$,
   $\hat{q_{0}} = (q_{0}, q_{0}')$,
   $\hat{\delta}((q,q'),\isymb) = (\delta(q,\isymb), \delta'(q',(\isymb,\lambda(q,\isymb))))$, and
   $\hat{\lambda}((q,q'),\isymb) = \lambda'(q',(\isymb,\lambda(q, \isymb)))$.
\fi
   
\subsubsection*{Multi-agent reactive systems}
A \emph{multi-agent reactive system} $\design$ is a tuple $(\proc, \ialphabet,  \oalphabet)$, where $\proc = \{\proc_1,\ldots,\proc_n\}$ is a set of agents, where each $\proc_i = (\states_i, q_{0,i},\ialphabeti,\oalphabeti,\delta_i,\lambda_i)$ is a reactive system.
%
While multiple agents may be able to read the same input variables to indicate
broadcast from the environment, the sets of outputs are pairwise disjoint: for $i \neq j$, we have $\out_i \cap \out_j = \emptyset$. Furthermore, agents cannot directly read each others outputs, that is, for all $i$ and $j$, we have $\out_i \cap \inp_j = \emptyset$. The outputs of the multiagent system  $\design$ are $\out = \bigcup_{i=1}^n \out_i$, and its inputs are $\inp  = \bigcup_{i=1}^n \inp_i$.
%
The joint behaviour of the multi-agent system is a reactive system 
$\design=(\states, q_{0}, \ialphabet,\oalphabet, \delta,
\lambda)$ defined as follows: the set $\states = \bigotimes_i \states_i$ of states is formed by the product of
the states of all agents $\proc_i \in \proc$. The initial state $q_{0}$ is formed
by the initial states $q_{0,i}$ of all $\proc_i \in \proc$. The transition function $\delta$ updates, for each agent $\proc_i \in \proc$, the $\states_i$ part of the state in accordance with the transition function $\delta_i$, using
the projection $\symb(I_i)$ as input. The output function $\lambda$ labels each state with the
union of the outputs of all $\proc_i \in \proc$ according to $\lambda_i$.



\subsubsection*{Specifications}

A \emph{specification} $\spec$ defines a set $\lang(\spec) \subseteq \alphabet^\infty$ of allowed traces.
%
A reactive system $\design$ \emph{realizes} $\spec$, denoted by $\design \models \spec$, iff
$\lang(\design) \subseteq \lang(\spec)$.
Given a set of propositions $\mathsf{AP}$, a formula in \emph{linear temporal logic} (LTL) describes a language in $(2^\mathsf{AP})^\omega$. LTL extends Boolean logic by the introduction of temporal operators such as $\LTLnext$ (next time), $\LTLglobally$ (globally), $\LTLfinally$ (eventually), and $\LTLuntil$ (until)~\cite{Pnueli77}.
%
$\varphi$ is called a \emph{safety specification} if every trace $\trace$ that is not in  $\lang(\spec)$  has a prefix $\tau$ such that all words starting with $\tau$ are also not in the language $\lang(\spec)$.
We represent a safety specification $\varphi$ by a safety automaton
$\varphi = (Q, q_0, \alphabet, \delta, F)$, where $F\subseteq Q$ is a set of safe states.

\subsubsection*{Games}
%
A game is a tuple $\game = (\gstates,
\ginit, \alphabet, \delta, \Acc, \Val)$,
where $\gstates$ is a finite set of states, $\ginit \in \gstates$ is the initial state,
$\delta: \gstates \times \alphabet \rightarrow \gstates$
is a complete transition function, $\Acc: (\gstates \times \alphabet \times \gstates)^\omega \rightarrow \bools$ is a winning condition
and defines the qualitative objective of the game, and $\Val: (\gstates \times \alphabet \times \gstates)^\omega \rightarrow \mathbb{R} \cup \{-\infty, \infty \}$ is a value function defining the quantitative objective of the game. A game can have a winning condition, a value function, or both.
%
The game is played  by two players:  the system and the environment. In every state $g\in \gstates$
(starting with $\ginit$), the environment chooses an input
$\isymb \in \ialphabet$, and then the system chooses some output $\osymb \in \oalphabet$. These choices define the next state $g' = \delta(g,(\isymb, \osymb))$, and so on. The resulting (infinite)
sequence $\overline{\pi} = (g_0,\isymb, \osymb, g_1) (g_1,\isymb, \osymb, g_2) \ldots$ is called a \emph{play}.
A deterministic  \emph{strategy} for the environment is a function
$\rho_e: \gstates^* \rightarrow \ialphabet$.
A nondeterministic (deterministic) \emph{strategy} for the system is a relation $\rho_s:
\gstates^* \times \ialphabet \rightarrow 2^{\oalphabet}$  (function $\rho_s:
\gstates^* \times \ialphabet \rightarrow \oalphabet$).

A play $\overline{\pi}$ is \emph{won} by the system iff $\Acc(\overline{\pi})=\top$.
 A strategy is \emph{winning} for the system if all plays $\overline{\pi}$ that can be
constructed when defining the outputs using the strategy result in
$\Acc(\overline{\pi})=\top$. The \emph{winning region} $\Win$ is the set of states
from which a winning strategy exists.
A \emph{permissive} winning strategy  $\rho_s:
\gstates^* \times \ialphabet \rightarrow 2^{\oalphabet}$ is a strategy that is not only winning for the system, but also contains all deterministic winning strategies.

A \emph{safety game} defines $\Acc$ via a set $F\subseteq \gstates$ of
safe states: $\Acc(\overline{\pi})=\top$ iff $g_i \in F$ for all $i \geq 0$, i.e., if only safe states are visited in the play $\overline{\pi}$. Otherwise, $\Acc(\overline{\pi})=\bot$.
%If a safety game is won by the system player, then there exists a permissive strategy $\rho_s$ that is \emph{memoryless}, i.e., has the form $\rho_s:
%\gstates \times \ialphabet \rightarrow 2^{\oalphabet}$.
%
The \emph{\buchi} winning condition is $\Acc(\overline{\pi})=\top$ iff $\inf(\overline{\pi})
\cap F \neq \emptyset$, where $F \subseteq G$ is
the set of accepting states and $\inf(\overline{\pi})$ is the set of states that occur infinitely often in $\overline{\pi}$.
We abbreviate the \buchi condition as $\mathcal{B}(F)$.
A \emph{Generalized Reactivity 1} (GR(1))
acceptance condition is a predicate $\bigwedge_{i=1}^{m} \mathcal{B}(E_{i}) \rightarrow \bigwedge_{i=1}^{n} \mathcal{B}(F_{i})$, with
$E_i \subseteq G$ and $F_i \subseteq G$.
 A \emph{Streett}
acceptance condition 
%with $k$ pairs 
is 
 $\bigwedge_{i=1}^{k} \mathcal{B}(E_{i}) \rightarrow \mathcal{B}(F_{i})$.

The quantitative objective of the system is to minimize $\Val(\overline{\pi})$, while the environment
 tries to maximize it.\looseness=-1
%
\iffalse MEAN-PAYOFF
A \emph{mean-payoff game} is a game where $\Val$ is defined via an edge labeling function $r :
\delta \rightarrow \{-W,\dots, W\}$, which assigns values between $-W$ and $W$ to edges. For a play $\pi =
e_0 e_1 e_2 \dots \in \delta^\omega, \Val(\overline{\pi}) = \lim \sup_{n\rightarrow\infty} \frac{1}{n+1} \sum_{i=0}^{n} r(e_i)$.
\fi


\subsubsection*{Properties of traces}
A finite trace $\trace
\in \alphabet^*$ is \emph{wrong} w.r.t. a specification $\varphi$, if the corresponding play cannot be won,
i.e., if there is no way for the system to guarantee that any
extension of $\trace$ satisfies $\spec$.
An output $\osymb$
is called \emph{wrong} for a trace $\trace$ and input $\isymb$, if it makes the trace wrong, i.e. when $\trace$ is not wrong, but $\trace \cdot (\isymb,\osymb)$ is. Given a sequence $(\itrace\parallel\otrace\parallel\otrace') \in (\ialphabet\times\oalphabet\times\oalphabet)^\infty$, we denote with $\widx(\itrace\parallel\otrace\parallel\otrace')$ the positions of occurrences of wrong outputs in $\otrace$. Formally, $i \in \widx(\itrace\parallel\otrace\parallel\otrace')$ iff  $\otrace[i]$ is wrong for the trace
$(\itrace[0,i)\parallel\otrace'[0,i))$ and the input $\itrace[i]$.

%
We denote with $\Proc = \{1,\ldots,n\}$ the set of agent ids of a multi-agent system $\design$.
%
For a set $\Pi \subseteq \Proc$, we define $O_{\Pi} = \bigcup_{i \in \Pi} O_i$ and $\alphabet_{O_{\Pi}} = 2^{O_{\Pi}}$. For $\osymb \in \oalphabet$ and $i \in \Proc$, we denote with $\osymb(O_i)$ the projection of $\osymb$ on $O_i$. For $\Pi \subseteq \Proc$, we define $\osymb(O_{\Pi})$ similarly. 

For  $\osymb,  \osymb' \in \oalphabet$, the set $\Diff(\osymb,\osymb') = \{i \in \Proc \mid \osymb(O_i) \neq \osymb'(O_i)\}$ gives the set of agents whose outputs in $\osymb$ differ from those in $\osymb'$.
Let $(\otrace \parallel \otrace')\in (\oalphabet\times\oalphabet)^\infty$ be a sequence of output pairs.
We call $(\otrace \parallel\otrace')$ a \emph{deviation period} if (1) $\otrace[i] \neq \otrace'[i ]$ for every $i < |\otrace|$  and (2) if $|\otrace| < \infty$, then $\otrace[|\otrace|] = \otrace'[|\otrace|]$.
Thus, a deviation period is either a finite sequence $(\otrace\parallel\otrace')$  consisting of differing outputs followed by a last letter where the two outputs agree, or an infinite sequence $(\otrace\parallel\otrace')$ where the outputs always differ. \looseness=-1
%We denote with $\Deviations(\otrace||\otrace')$ the set of all deviation periods of $(\otrace||\otrace')$.




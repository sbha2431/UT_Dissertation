%We have provided a systematic, and general, approach to solving surveillance problems for autonomous agents. The framework we present extends previous work by including additional sensor modalities such as static sensors and accounting for imperfect sensors. Furthermore, we show the viability of the framework by simulating the execution of the synthesized surveillance strategies in realistic simulation environments against human controlled adversaries. Some directions of future work that we are currently exploring include the following:
%
%\begin{itemize}
%
%\item allowing for false positives to occur in target state estimation;
%\item automated generation of the initial abstraction based on features of the map of the surveillance area;
%    \item implementation and execution of the synthesized surveillance strategies on actual hardware;
%    \item considering different classes of targets, for example, both hostile and non-hostile targets;
%    \item learning the behaviour of a target over time, to allow the agent to improve its performance while still guaranteeing the satisfaction of the surveillance specification. 
%\end{itemize}



We have presented a novel approach to solving a surveillance problem with information guarantees. We provided a framework that enables the  formalization of the surveillance synthesis problem as a two-player, partial-information game. We then presented a method to reason over the belief that the agent has over the target's location and specify formal surveillance requirements. The user can tailor the behaviour to their specific application by using a combination of safety and liveness surveillance objectives. Furthermore, we show the viability of the framework by simulating the execution of the synthesized surveillance strategies in realistic simulation environments as well as on hardware against human controlled adversaries.

The benefit of the proposed framework is that it allows it leverages techniques successfully used in verification and reactive synthesis to develop efficient methods for solving the surveillance problem. There are several promising  avenues of future work using and extending this framework. Some of which currently being explored are the following;
\begin{itemize}

\item allowing for false positives to occur in target state estimation;
\item automated generation of the initial abstraction based on features of the map of the surveillance area;
    \item considering different classes of targets, for example, both hostile and non-hostile targets;
    \item learning the behaviour of a target over time, to allow the agent to improve its performance while still guaranteeing the satisfaction of the surveillance specification. 
\end{itemize}
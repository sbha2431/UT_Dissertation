As we have shown, the iterative abstraction refinement procedure we have described is guaranteed to terminate with either a winning strategy for the agent or with a concrete counterexample. In theory, a user can start with a very uninformed abstraction (for example, a trivial abstraction with one abstraction partition consisting of the entire state space), and the counterexample-guided refinement procedure will eventually terminate with an abstraction and a corresponding strategy. In practice, however, it is advisable to start with a suitably chosen informed abstraction, in order to reduce the number of necessary refinement iterations. In the next section we show some examples of user-provided abstractions which even require no refinement at all. Furthermore, user-provided abstractions are oftentimes 
more `intuitive' compared to the ones generated by the automated refinement procedure.

Ideally, constructing a suitable initial abstraction should be automated as well. However, abstractions are highly dependent on the sensor function and the structure of the environment. The process of generating initial abstractions based on automatically identified map features is on-going work. 
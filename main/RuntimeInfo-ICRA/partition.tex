Let $\{\overline{p}_i\in \mathcal{P}\}^{N}_{i=1}$ be the given candidate instantiations of the runtime information, and let 
$\{{\rho_\out^\ast}_i\in \mathcal{M}_\out\}_{i=1}^{N}$ be the corresponding optimal strategies. That is, for each $i$, ${\rho_\out}_i^\ast$ is an optimal solution of \eqref{prob:opt} for $\overline{p}_i$. 
Under Assumption~\ref{assum:C_lin} below, we can compute a collection of polytopes ${\{\mathcal{S}_i\}}_{i=1}^N$ such that ${\rho_\out^\ast}_i$ is $\epsilon$-optimal, whenever $\overline p \in \mathcal{S}_i$.
\subsubsection*{Assumption}\label{assum:C_lin}
  The cost function $C: \mathcal M_\out\times \mathcal{P}
  \to \mathbb{R}$ is such that for all $\rho_\out \in \mathcal M_\out$ and $\overline p = {[q_1\ q_2\ \ldots\ q_n]}^\top \in \mathcal P \subseteq \mathbb{R}^n$:
  \begin{align}
        C(\rho_\out, \overline{p}) = \sum_{i=1}^n
        C(\rho_\out, \overline{e}_i)\, q_i,\label{eq:V_lin_decompose}
    \end{align}
    where $\overline{e}_i$ is the vector that has $1$ at
     position $i$ and $0$ elsewhere.


We address Problem~\ref{prob_st:main} by defining a collection of polytopes $\{\mathcal{S}_i\}_{i=1}^N$, where $\mathcal{S}_i= \{ \overline{p} \in \mathbb{R}^n \mid H_i \overline{p}\leq
\overline{b}_i\}$ and
{\small\begin{align}
    H_i &= \left[
        \begin{array}{ccc}
            C({\rho_\out}_i^\ast, \overline{e}_1) -
            C({\rho_\out}_1^\ast, \overline{e}_1) &
            \cdots &
            C({\rho_\out}_i^\ast, \overline{e}_n) -
            C({\rho_\out}_1^\ast, \overline{e}_n)\\
            C({\rho_\out}_i^\ast, \overline{e}_1) -
            C({\rho_\out}_2^\ast, \overline{e}_1) &
            \cdots &
            C({\rho_\out}_i^\ast, \overline{e}_n) -
            C({\rho_\out}_2^\ast, \overline{e}_n)\\
            \vdots & \vdots & \vdots \\
            C({\rho_\out}_i^\ast, \overline{e}_1) -
            C({\rho_\out}_N^\ast, \overline{e}_1) &
            \cdots &
            C({\rho_\out}_i^\ast, \overline{e}_n) -
            C({\rho_\out}_N^\ast, \overline{e}_n)\\
            C({\rho_\out}_i^\ast, \overline{e}_1) -
            C^\ast(\overline{e}_1) &
            \cdots &
            C({\rho_\out}_i^\ast, \overline{e}_n) -
            C^\ast(\overline{e}_n)
    \end{array}\right] \label{eq:A_defn} \\
        \overline{b}_i&={[0\ 0\ \ldots\ 0\
        \epsilon]}^\top\in \mathbb{R}^{N+1}\label{eq:b_defn}
\end{align}}%\normalsize%
with $H_i\in \mathbb{R}^{(N+1)\times n}$. By \eqref{prob:opt} and Assumption~\ref{assum:C_lin}, we have the following upper bound on $C^\ast(\overline{p})$ for any runtime information vector $ \overline{p}={[q_1\ q_2\ \ldots\ q_n]}^\top\in
\mathcal{P}$,
\begin{align}
    C^\ast( \overline{p})\leq
    C({\rho_\out}_i^\ast,
    \overline{p})=\sum_{j=1}^nC({\rho_\out}_i^\ast,
    \overline{e}_j)q_j\label{eq:C_ub},\quad \forall
    i\in\{1,\ldots,N\}.
\end{align}

Theorem~\ref{thm:bounds} below ensures that using ${\rho_\out^\ast}_i$ for vectors $\overline p \in \mathcal{S}_i$ results in $\epsilon$-optimal performance, provided the polytopes $\mathcal{S}_i$ are non-empty. In other words, the difference between $C^\ast(\overline p)$, the optimal performance for a runtime information $\overline{p}$,  and $C(\rho_{\out_i}^\ast, \overline p)$, the attained performance due to choice of strategy $\rho_{\out_i}^\ast$,  can not be larger than $\epsilon$, whenever $\overline p \in \mathcal{S}_i$. To complete the discussion, we provide a sufficient condition for non-empty polytopes $\mathcal{S}_i$ in Proposition~\ref{prop:nonempty}.
 
\begin{thm}\label{thm:bounds}
    Let the polytopes $\mathcal{S}_i$ be non-empty. Given runtime information $ \overline{p} \in \mathbb{R}^n$,
    if $ \overline{p}\in \mathcal{S}_i$ for some
    $i\in\{1,\ldots, N\}$, then
    \begin{align}
        C({\rho_\out}_i^\ast,
    \overline{p}) - \epsilon \leq
        C^\ast(\overline{p}) \leq
        C({\rho_\out}_i^\ast,
    \overline{p}). \nonumber
    \end{align}
\end{thm}
\begin{proof}
   The upper bound on $C^\ast( \overline{p})$ follows from
   \eqref{eq:C_ub}. Consider the collection of polytopes $
    \mathcal{T}_i$ constructed using the first $N$
    hyperplanes in \eqref{eq:A_defn} and \eqref{eq:b_defn}.
    For every $ \overline{p}\in \mathcal{T}_i$,
    % ={[q_1\ %q_2\ \ldots\ q_n]}^\top\in \mathcal{T}_i\subset \mathbb{R}^n$,
    \begin{align}
        %\sum_{j=1}^nC(\overline{\gamma}_i^\ast,
        %\overline{e}_j) q_j&\leq
        %\sum_{j=1}^nC(\overline{\gamma}_k^\ast,
        %\overline{e}_j)q_j,\ \forall
        %k\in\{1,\ldots,N\}\setminus\{i\}\label{eq:dom}.
        C({\rho_\out}_i^\ast, \overline{p})&\leq
        C({\rho_\out}_k^\ast, \overline{p}),\ \forall
        k\in\{1,\ldots,N\}\setminus\{i\}\label{eq:dom}.
    \end{align}
    In other words, among the $N$  strategies
    ${\rho_\out}_{(\cdot)}^\ast$,
    ${\rho_\out}_{i}^\ast$ provides the tightest upper
    bound on $C^\ast( \overline{p})$ due to \eqref{eq:C_ub}. 
    
   
   We next prove the lower bound. The last
   hyperplane in \eqref{eq:A_defn} and \eqref{eq:b_defn}
   guarantees that $ C(
   {\rho_\out}_i^\ast, \overline{p}) -  \overline{\ell}^\top \overline{p} \leq \epsilon$ for
   every $ \overline{p}\in \mathcal{S}_i$, where $\overline{\ell}_j = C^\ast( \overline{e}_j)$. On adding and
   subtracting $C^\ast( \overline{p})$, we have
   $C({\rho_\out}_i^\ast, \overline{p}) - C^\ast(
   \overline{p}) + C^\ast(\overline{p}) -
   \overline{\ell}^\top \overline{p} \leq \epsilon$. Since
   $C( \cdot, \overline{e}_j)\geq \ell_j$ by definition of $
   \overline{\ell}$, we have $C^\ast(\overline{p}) -
   \overline{\ell}^\top \overline{p} \geq 0$ for every $
   \overline{p}\in \mathcal{S}_j$. Therefore, $C({\rho_\out}_i^\ast, \overline{p}) - C^\ast(
   \overline{p}) \leq \epsilon$.
\end{proof}
\begin{prop}\label{prop:nonempty}
    Let Assumption~\ref{assum:C_lin} hold. For every $i=1,\ldots,N$ the polytope $\mathcal{S}_i$ is non-empty, provided that $\epsilon \geq \max_i\left\{ C( {\rho_\out}_i^\ast, \overline{p}_i) -
    \overline{\ell}^\top \overline{p}_i\right\}$. Here,
    $\overline{\ell}={[C^\ast( \overline{e}_1)\ C^\ast(\overline{e}_2)\ \ldots\ C^\ast( \overline{e}_n)]}^\top\in \mathbb{R}^n$.
    % Moreover, each of the pairwise-intersections of the polytopes is either a hyperplane or is the empty set.
\end{prop}
\begin{proof}
    The polytopes $ \mathcal{T}_i$ (defined in the proof of Theorem~\ref{thm:bounds}) are non-empty, since they
    contain $ \overline{p}_i$ by the optimality of $
    {\rho_\out}_i^\ast$ in \eqref{prob:opt}.
    The last hyperplane in \eqref{eq:A_defn} and
    \eqref{eq:b_defn} is also satisfied by $
    \overline{p}_i$, thanks to the use of $\epsilon$ in $\overline b_i$.
    Thus, its intersection with $ \mathcal{T}_i$, which
    yields the polytope $ \mathcal{S}_i$, is non-empty.
    % By construction, the
    % pairwise-intersection of the polytopes $ \mathcal{T}_i$
    % are either hyperplanes (overlapping boundary of
    % $\mathcal{T}_i$ and $\mathcal{T}_k,\ k\neq i$) or empty. 
\end{proof}

\begin{abstract}
Urban air mobility (UAM) refers to air transportation services within an urban area, often in an on-demand fashion. We study air traffic management (ATM) for vehicles in a UAM fleet, while guaranteeing system safety requirements such as traffic separation. Existing ATM methods for unmanned aerial systems such as UTM utilize alternative approaches which do not provide strict safety guarantees. No established infrastructure exists for providing ATM at scale for UAM. We provide a decentralized, hierarchical approach for UAM ATM that allows for scalability to high traffic densities as well as providing \emph{theoretical guarantees of correctness} with respect to user provided safety specifications. Our main contributions are two-fold. First, we propose a novel UAM ATM architecture that divides control authority between \emph{vertihubs} that are each in charge of all UAM vehicles in their local airspace. Each vertihub also contains a number of \emph{vertiports} that are in charge of UAM vehicle takeoffs and landings. The resulting architecture is decentralized and hierarchical, which not only enables scalability, but also robustness in the event of any individual vertihub or vertiport no longer being operational. Second, we provide a contract-based correct-by-construction reactive synthesis approach that provably guarantees safety properties with respect to user-provided safety specifications in linear temporal logic. We demonstrate the approach on large-volume UAM air traffic data. 


%\keywords{reactive synthesis \and system safety \and air traffic management}
\end{abstract}

\chapter{Conclusions and Future Work}
The focus of this dissertation has been on assuring \emph{outcomes}, that is, it aims to answer the question \emph{how do we guarantee that this autonomous agent achieve the desired behavior?} The main antagonist is, of course, scalability. Specifically, scalability issues arising from the problems of partial-information (Part I) and multiple agents (Part II). This dissertation then provides methods to address \emph{scalable assurance} in both these settings by employing belief-abstraction and decentralization. 

\section{Summary}
Part I focused on \emph{design-time synthesis}. In particular,Chapter 2 provided a belief-abstraction based technique for synthesis from specification in partial-information settings against adversarial agents. However, a main drawback of design-time approaches is that it cannot respond to changes in the environment or learn to behave more optimally without re-synthesizing from scratch. To address this weakness, Chapter 3 presented an approach where multiple strategies were synthesized at design-time and the agent can switch between them at runtime in response to information observed from the environment at runtime. Each of the strategies were guaranteed to be correct with respect to the specification, and by switching between them at runtime, the approach can guarantee both correctness as well as $\epsilon-$optimality. 

Part II focused on \emph{runtime enforcement} where the onus is on guaranteeing safety without full knowledge of the underlying system's goals and design. Specifically, the focus is on multi-agent systems. Previous methods for guaranteeing safety at runtime could not address the needs of a potentially independently designed and operated multi-agent system. In particular, previous methods implicitly assumed that safety violations at runtime were a result of system error and not a wrong action. Chapter 4 studied the problem where the \emph{intent} of each of the individual agents violates global safety requirements. The chapter then presents an approach were safety can be guaranteed while minimizing a cost function for interference which can be provided based on the application need. However, the approach in Chapter 4 was centralized which assumed the presence of a single entity than can observe the entire state space as well as control all agents which is unrealistic in practice and intractable for larger number of agents. Chapter 5 addresses this weakness by creating on-board runtime enforcement modules for each agent and guaranteed safety and fairness for the entire system. The key assumption however, is that only \emph{localized safety} can be guaranteed. 

Finally, in Part III, the dissertation combines the techniques in Parts II and III for the application area of urban air mobility. Chapter 6 introduces a novel hierarchical, deceentralized synthesis architecture to enabled scalable assurance for air traffic management in urban air mobility. Specifically, it employs design-time synthesis to generate strategies for vertihubs that then act as runtime enforcers for the multi-agent system of air vehicles. However, the approach is purely qualitative, and if a feasible strategy does not exist, then no controllers are synthesized. Chapter 7 addresses this limitation by introducing \emph{minimum-violation synthesis} amongst the vertihubs that allows for them to violate specifications if no satisfying trace exists. 

\section{Limitations and future work}

\subsection{Integrating empirical testing}
The approach taken in this dissertation as a whole has been to separate empirical testing from theoretical assurance. More explicitly, we first provide theoretical guarantees that the system is correct with respect to the given specifications, and then test the system empirically (either in simulation or limited hardware) to demonstrate viability. However, an interesting avenue for future research is to more directly integrate the two approaches. Much of the work presented in this dissertation relies on a relatively strong assumption on the knowledge of the model. Empricial testing provides a way to generate counterexamples for incorrect model assumptions which can then be used to adjust theoretical guarantees. A systematic approach to integrating empirical testing with theoretical assurance is a promising direction of future research. 

\subsection{Characterizing infrastructure and capabilities}
The dissertation focuses on the capabilities and infrastructure of the system as problem inputs and then provides a method to generate provably correct controllers. However, in applications such as urban air mobility, the infrastrcuture and capabilities of the system are still in its infancy. Such a setting is a golden opprotunity to build infrastructure from the ground up in which assurance and other theoretical guarantees can be provided. Indeed, the work in this dissertation can be naturally be applied backwards. Starting from the required theoretical guarantees, we can ask the question \emph{what is the required infrastructure, communication and sensing capabilities needed to provide the required guarantees?}. Many mature applications (self-driving cars being one) have to fit assurance around existing infrastructure (such as roads and traffic lights). UAM offers a promising research direction of allowing the infrastructure to be designed around safety assurance in mind. 
\chapter{Conclusions and Future Work}
The focus of this dissertation has been on assuring \emph{outcomes}, that is, it aims to answer the question \emph{how do we guarantee that this autonomous agent achieve the desired behavior?} The main antagonist is, of course, scalability. Specifically, scalability issues arising from the problems of partial-information (Part II) and multiple agents (Part III). This dissertation then provides methods to address \emph{scalable assurance} in both these settings by employing belief-abstraction and decentralization. 

\section{Summary}
Part II focused on \emph{design-time synthesis}. In particular,Chapter 2 provided a belief-abstraction based technique for synthesis from specification in partial-information settings against adversarial agents. However, a main drawback of design-time approaches is that it cannot respond to changes in the environment or learn to behave more optimally without re-synthesizing from scratch. To address this weakness, Chapter 3 presented an approach where multiple strategies were synthesized at design-time and the agent can switch between them at runtime in response to information observed from the environment at runtime. Each of the strategies were guaranteed to be correct with respect to the specification, and by switching between them at runtime, the approach can guarantee both correctness as well as $\epsilon-$optimality. 

Part III focused on \emph{runtime enforcement} where the onus is on guaranteeing safety without full knowledge of the underlying system's goals and design. Specifically, the focus is on multi-agent systems. Previous methods for guaranteeing safety at runtime could not address the needs of a potentially independently designed and operated multi-agent system. In particular, previous methods implicitly assumed that safety violations at runtime were a result of system error and not a wrong action. Chapter 4 studied the problem where the \emph{intent} of each of the individual agents violates global safety requirements. The chapter then presents an approach were safety can be guaranteed while minimizing a cost function for interference which can be provided based on the application need. However, the approach in Chapter 4 was centralized which assumed the presence of a single entity than can observe the entire state space as well as control all agents which is unrealistic in practice and intractable for larger number of agents. Chapter 5 addresses this weakness by creating on-board runtime enforcement modules for each agent and guaranteed safety and fairness for the entire system. The key assumption however, is that only \emph{localized safety} can be guaranteed. 

Finally, in Part IV, the dissertation combines the techniques in Parts II and III for the application area of urban air mobility. Chapter 6 introduces a novel hierarchical, deceentralized synthesis architecture to enabled scalable assurance for air traffic management in urban air mobility. Specifically, it employs design-time synthesis to generate strategies for vertihubs that then act as runtime enforcers for the multi-agent system of air vehicles. However, the approach is purely qualitative, and if a feasible strategy does not exist, then no controllers are synthesized. Chapter 7 addresses this limitation by introducing \emph{minimum-violation synthesis} amongst the vertihubs that allows for them to violate specifications if no satisfying trace exists. 

\section{Limitations and future work}
The dissertation thus far has  built a foundation on which to address theoretical challenges for assured decision-making in autonomy. This section will adress future work in three possible directions with impacts in the short, medium and long terms respectively that are designed to inform and build upon each other. 

In the short-term, it is necessary to continue to advance the state-of-the-art in \emph{scalable verification} for autonomy by incorporating real-world challenges such as partial-information and unreliable communication. These advancements will have immediate impact in developing new business models for emerging technologies. Over the medium-term, autonomous systems will likely be used across many fields and thus there can be no one-size-fits-all approach to assurance.  Hence, it is important to study how to rigorously characterize assurance principles and outcomes in order to build generalized \emph{context-aware} assurance methods that can incorporate requirements specific to each operating environment. Finally, the long-term broader societal impacts of the increasing integration of autonomy cannot be ignored. Autonomy, while having significant potential for societal benefits, also can lead to disproportionate outcomes for different communities. Research on \emph{societal fairness of autonomy} in the long term is crucial to the long-term health of society.

\subsection{Expand technical scope}
The work in this dissertation has introduced and explored scalable techniques for assured decision-making. Moving forward, we must continue to expand technical capabilities by incorporating immediate real-world concerns such as privacy, certification from regulators, and sensing and communication restrictions that are currently major road blocks in scalable assured autonomy. Questions like \emph{can we formally characterize the minimum communication requirements in an autonomous vehicle fleet to guarantee safety and mission completion?}, \emph{can we verify safety for a system when not all agents can be sensed all times?}, and \emph{can we protect privacy amongst multiple fleet operators with competing priorities?}. This dissertation has laid the foundation to answer these questions in the short-term by studying partial-information and multi-agent systems. However, it has not studied the two in combination and there lies an immediately impactful avenue for future research. 

Another important direction in expanding technical scope is incorporating more physical experiments. This dissertation focuses heavily on building a theoretical foundation, but there is still much to be learned in seeing where the assumptions fail in practice. In particular, the techniques presented in this dissertation can be used to tailor physical experiments in order to better validate the theory. 

\subsection{Expand contextual scope}
In the short-term, research must broaden the \emph{technical} scope of assured autonomy. Looking further forward, it is clear that autonomous systems will only grow ever more pervasive across all fields and disciplines and the need for assurance will grow stronger. It is thus necessary to broaden the \emph{contextual} scope as well.  Hence, the creation and compilation of a set of assurance principles and outcomes to build a \emph{generalized framework of assured autonomy} in order to provide a systematic approach to formal assurance will allow subject matter experts to fully unlock the power of autonomous systems in their respective domains. For example, autonomous cars and autonomous air vehicles operate in different environments and have different assurance concerns. However, having a general framework that allows for context-aware assurance with input from the relevant experts can accelerate the process of innovation in both areas. 

Characterizing general principles of assurance across domains can empower users who are not experts in assurance to innovate with guarantees. While this dissertation touches on the concept of allowing for non-expert interaction (e.g., with the use of temporal logics instead of reward functions), there is much work to be done in incorporating more natural-language in the specifying of requirements. 

\subsection{Investigate societal impact}
It is contingent upon all researchers of autonomous systems to study the broader societal impacts of increasing integration of autonomy. Specifically, how can we avoid disproportionate outcomes for different communities? We are ethically obliged to study \emph{societal fairness of autonomy} in order to quantify the impacts of increasing autonomy by applying the general principles of assurance typically used for safety or performance. %Such a study naturally requires an inter-disciplinary approach in order to 

One of the main issues of such a long-term view is that impacts of design decisions and processes may not manifest until the scale of operations grows. Thus, research on how to mitigate societal impact is challenging as the impacts can be delayed, or difficult to attribute. To address this difficutly, another aspect to consider in building any assurance framework is \emph{agility}. Specifically, \emph{how do we integrate observed knowledge of societal impacts into any existing framework?}. More specifically, how can we build a system with \emph{responsiveness} in mind. Unknown unknowns cannot be predicted, however, systems should be designed to be agile and responsive to emerging information. How to quantify such a concept is challenging and what makes one system more or less responsive than another is an open question.   

\subsection{Integrate empirical testing}
The approach taken in this dissertation as a whole has been to separate empirical testing from theoretical assurance. More explicitly, we first provide theoretical guarantees that the system is correct with respect to the given specifications, and then test the system empirically (either in simulation or limited hardware) to demonstrate viability. However, an interesting avenue for future research is to more directly integrate the two approaches. Much of the work presented in this dissertation relies on a relatively strong assumption on the knowledge of the model. Empircal testing provides a way to generate counterexamples for incorrect model assumptions which can then be used to adjust theoretical guarantees. A systematic approach to integrating empirical testing with theoretical assurance is a promising direction of future research. 

Integrating empirical testing can impact the broadening of both technical and contextual scopes. More specifically, understanding exactly how theoretical assumptions fail in real-world testing will allow for better provable guarantees. The direction of empirical studies can also be focused to better validate theoretical assumptions. More broadly, formalizing the relationship between theoretical assurances and experimental validation can generalize across domains and lead to a broader application of assurance techniques. 

\subsection{Characterize infrastructure and capabilities}
The dissertation focuses on the capabilities and infrastructure of the system as problem inputs and then provides a method to generate provably correct controllers. However, in applications such as urban air mobility, the infrastrcuture and capabilities of the system are still in its infancy. Such a setting is a golden opprotunity to build infrastructure from the ground up in which assurance and other theoretical guarantees can be provided. Indeed, the work in this dissertation can be naturally be applied backwards. Starting from the required theoretical guarantees, we can ask the question \emph{what is the required infrastructure, communication and sensing capabilities needed to provide the required guarantees?}. Many mature applications (self-driving cars being one) have to fit assurance around existing infrastructure (such as roads and traffic lights). UAM offers a promising research direction of allowing the infrastructure to be designed around safety assurance in mind. 

